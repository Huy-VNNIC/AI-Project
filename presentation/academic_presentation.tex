\documentclass[aspectratio=169,12pt]{beamer}
\usetheme{Madrid}
\usecolortheme{default}

% Packages
\usepackage[utf8]{inputenc}
\usepackage{amsmath}
\usepackage{amssymb}
\usepackage{graphicx}
\usepackage{booktabs}
\usepackage{xcolor}
\usepackage{tikz}
\usepackage{subcaption}

% Custom colors
\definecolor{primaryblue}{RGB}{52,152,219}
\definecolor{secondarygreen}{RGB}{39,174,96}
\definecolor{accentorange}{RGB}{243,156,18}
\definecolor{alertred}{RGB}{231,76,60}

% Theme customization
\setbeamercolor{structure}{fg=primaryblue}
\setbeamercolor{block title}{bg=primaryblue,fg=white}
\setbeamercolor{block body}{bg=primaryblue!10}

% Footer
\setbeamertemplate{footline}{
  \leavevmode%
  \hbox{%
  \begin{beamercolorbox}[wd=.33\paperwidth,ht=2.25ex,dp=1ex,center]{author in head/foot}%
    \usebeamerfont{author in head/foot}\insertshortauthor
  \end{beamercolorbox}%
  \begin{beamercolorbox}[wd=.34\paperwidth,ht=2.25ex,dp=1ex,center]{title in head/foot}%
    \usebeamerfont{title in head/foot}\insertshorttitle
  \end{beamercolorbox}%
  \begin{beamercolorbox}[wd=.33\paperwidth,ht=2.25ex,dp=1ex,right]{date in head/foot}%
    \usebeamerfont{date in head/foot}\insertshortdate{}\hspace*{2em}
    \insertframenumber{} / \inserttotalframenumber\hspace*{2ex} 
  \end{beamercolorbox}}%
  \vskip0pt%
}

% Title information
\title[AI-Based Software Effort Estimation]{Multi-Schema Software Effort Estimation Using Machine Learning}
\subtitle{An Enhanced COCOMO II Approach with Heterogeneous Data Integration}
\author[Phan Hoang Long]{Phan Hoang Long \\ \texttt{phanhoanglong@chungbuk.ac.kr}}
\institute[Chungbuk National University]{
  Department of Computer Science \\
  Chungbuk National University
}
\date{December 15, 2025}

\begin{document}

% ==================== SLIDE 1: Title ====================
\begin{frame}
  \titlepage
\end{frame}

% ==================== SLIDE 2: Motivation ====================
\begin{frame}{Motivation: Why Effort Estimation Matters}
  
  \begin{columns}[T]
    \begin{column}{0.48\textwidth}
      \textbf{Problem:}
      \begin{itemize}
        \item Inaccurate estimation $\rightarrow$ \textcolor{alertred}{Budget overrun}
        \item Traditional COCOMO II: \textcolor{alertred}{Limited adaptability}
        \item Real-world data: \textcolor{alertred}{Heterogeneous \& inconsistent}
      \end{itemize}
      
      \vspace{0.5cm}
      \textbf{Impact:}
      \begin{itemize}
        \item 85\% projects: Budget issues
        \item 78\% projects: Schedule delays
        \item 42\% risk of project failure
      \end{itemize}
    \end{column}
    
    \begin{column}{0.48\textwidth}
      \begin{figure}
        \includegraphics[width=\textwidth]{figures/fig1_problem_illustration.pdf}
        \caption*{\small Current challenges in effort estimation}
      \end{figure}
    \end{column}
  \end{columns}
  
  \vspace{0.3cm}
  \begin{alertblock}{Research Gap}
    \small \textit{Lack of unified framework to handle heterogeneous project data across different sizing schemas (LOC, FP, UCP)}
  \end{alertblock}
  
\end{frame}

% ==================== SLIDE 3: Research Contributions ====================
\begin{frame}{Research Contributions}
  
  \begin{block}{Main Contributions}
    \textbf{Unified multi-schema preprocessing pipeline + improved ML-based prediction}
  \end{block}
  
  \vspace{0.3cm}
  
  \begin{columns}[T]
    \begin{column}{0.32\textwidth}
      \begin{center}
        \colorbox{primaryblue!20}{\parbox{0.9\textwidth}{
          \centering
          \textbf{1. Data Integration} \\[0.2cm]
          \small Automatic normalization for LOC/FP/UCP schemas
        }}
      \end{center}
    \end{column}
    
    \begin{column}{0.32\textwidth}
      \begin{center}
        \colorbox{secondarygreen!20}{\parbox{0.9\textwidth}{
          \centering
          \textbf{2. ML Models} \\[0.2cm]
          \small Benchmark: COCOMO II vs LR/DT/RF/GB
        }}
      \end{center}
    \end{column}
    
    \begin{column}{0.32\textwidth}
      \begin{center}
        \colorbox{accentorange!20}{\parbox{0.9\textwidth}{
          \centering
          \textbf{3. Deployment} \\[0.2cm]
          \small REST API for real-world usage
        }}
      \end{center}
    \end{column}
  \end{columns}
  
  \vspace{0.5cm}
  
  \begin{itemize}
    \item \textbf{Dataset:} Integrated 320+ projects from multiple sources
    \item \textbf{Best Model:} Random Forest achieves \textcolor{secondarygreen}{\textbf{38\% MMRE}}, \textcolor{secondarygreen}{\textbf{58\% PRED(25)}}
    \item \textbf{Impact:} Reduces estimation error by \textcolor{secondarygreen}{\textbf{34\%}} vs COCOMO II baseline
  \end{itemize}
  
\end{frame}

% ==================== SLIDE 4: COCOMO II Background ====================
\begin{frame}{Background: COCOMO II Model}
  
  \begin{columns}[T]
    \begin{column}{0.55\textwidth}
      \textbf{Core Formula:}
      
      \begin{figure}
        \includegraphics[width=0.95\textwidth]{figures/fig10_cocomo_formula.pdf}
      \end{figure}
      
      \vspace{-0.3cm}
      \textbf{Key Parameters:}
      \begin{itemize}
        \item $A = 2.94$: Calibration constant
        \item $E$: Scale factor (precedentedness, flexibility, ...)
        \item $EM_i$: Effort multipliers (17 factors)
      \end{itemize}
    \end{column}
    
    \begin{column}{0.42\textwidth}
      \textbf{Output Derivation:}
      \begin{enumerate}
        \item \textbf{Effort} (person-months)
        \item \textbf{Duration}: $TDEV = C \times Effort^D$
        \item \textbf{Team Size}: $\frac{Effort}{Duration}$
      \end{enumerate}
      
      \vspace{0.5cm}
      \begin{exampleblock}{Limitation}
        \small Traditional COCOMO II assumes \textit{homogeneous data} and \textit{manual calibration}. Not scalable for modern diverse projects.
      \end{exampleblock}
    \end{column}
  \end{columns}
  
  \vspace{0.2cm}
  \footnotesize \textit{Reference: Boehm et al., Software Cost Estimation with COCOMO II (2000)}
  
\end{frame}

% ==================== SLIDE 5: Dataset Overview ====================
\begin{frame}{Data Sources \& Dataset Summary}
  
  \begin{columns}[T]
    \begin{column}{0.48\textwidth}
      \textbf{Data Sources:}
      \begin{itemize}
        \item \textbf{LOC:} NASA COCOMO, COC81
        \item \textbf{FP:} Desharnais, Albrecht
        \item \textbf{UCP:} Use Case Points
      \end{itemize}
      
      \vspace{0.3cm}
      \begin{table}[h]
        \centering
        \small
        \begin{tabular}{lcc}
          \toprule
          \textbf{Schema} & \textbf{N} & \textbf{Metric} \\
          \midrule
          LOC & 180 & KLOC \\
          FP & 95 & FP \\
          UCP & 45 & UCP \\
          \midrule
          \textbf{Total} & \textbf{320} & Mixed \\
          \bottomrule
        \end{tabular}
      \end{table}
    \end{column}
    
    \begin{column}{0.48\textwidth}
      \textbf{Key Challenge:}
      \begin{alertblock}{Data Heterogeneity}
        \small
        \begin{itemize}
          \item Different size metrics
          \item Inconsistent effort units
          \item Missing values \& contexts
        \end{itemize}
      \end{alertblock}
      
      \vspace{0.2cm}
      \begin{exampleblock}{Our Solution}
        \small Automatic \textbf{schema detection} + \textbf{unit normalization} pipeline
      \end{exampleblock}
    \end{column}
  \end{columns}
  
\end{frame}

% ==================== SLIDE 6: Data Heterogeneity Visualization ====================
\begin{frame}{Challenge: Data Heterogeneity}
  
  \begin{figure}
    \centering
    \includegraphics[width=0.85\textwidth]{figures/fig2_data_heterogeneity.pdf}
    \caption*{\textbf{Before:} Incompatible schemas $\rightarrow$ \textbf{After:} Unified normalization}
  \end{figure}
  
  \vspace{0.3cm}
  \begin{block}{Pipeline Handles:}
    \begin{itemize}
      \item \textbf{Unit conversion:} Hours $\rightarrow$ Person-Months, LOC $\rightarrow$ KLOC
      \item \textbf{Missing values:} IQR-based outlier detection, median imputation
      \item \textbf{Schema tagging:} Preserve origin for schema-specific modeling
    \end{itemize}
  \end{block}
  
\end{frame}

% ==================== SLIDE 7: Preprocessing Pipeline ====================
\begin{frame}{Preprocessing \& Normalization Pipeline}
  
  \begin{figure}
    \centering
    \includegraphics[width=0.75\textwidth]{figures/fig3_pipeline_flowchart.pdf}
    \caption*{\textbf{End-to-end automated preprocessing pipeline}}
  \end{figure}
  
  \vspace{-0.2cm}
  
  \begin{columns}[T]
    \begin{column}{0.48\textwidth}
      \textbf{Key Steps:}
      \begin{enumerate}
        \item Schema detection (LOC/FP/UCP)
        \item Unit standardization
        \item Outlier handling (IQR method)
      \end{enumerate}
    \end{column}
    
    \begin{column}{0.48\textwidth}
      \begin{enumerate}
        \setcounter{enumi}{3}
        \item log1p transformation (linearization)
        \item Feature scaling \& encoding
        \item Export ML-ready data
      \end{enumerate}
    \end{column}
  \end{columns}
  
\end{frame}

% ==================== SLIDE 8: Experimental Setup ====================
\begin{frame}{Models \& Experimental Setup}
  
  \begin{columns}[T]
    \begin{column}{0.48\textwidth}
      \textbf{Models Evaluated:}
      \begin{itemize}
        \item \textbf{Baseline:} COCOMO II (analytical)
        \item \textbf{ML Models:}
          \begin{itemize}
            \item Linear Regression (LR)
            \item Decision Tree (DT)
            \item \textcolor{secondarygreen}{\textbf{Random Forest (RF)}}
            \item Gradient Boosting (GB)
          \end{itemize}
      \end{itemize}
      
      \vspace{0.3cm}
      \textbf{Training Strategy:}
      \begin{itemize}
        \item 80/20 train-test split
        \item GridSearchCV for hyperparameters
        \item 5-fold cross-validation
      \end{itemize}
    \end{column}
    
    \begin{column}{0.48\textwidth}
      \textbf{Evaluation Metrics:}
      \begin{table}[h]
        \small
        \begin{tabular}{ll}
          \toprule
          \textbf{Metric} & \textbf{Preference} \\
          \midrule
          MAE & Lower $\downarrow$ \\
          RMSE & Lower $\downarrow$ \\
          MMRE & Lower $\downarrow$ \\
          PRED(25) & Higher $\uparrow$ \\
          R² & Higher $\uparrow$ \\
          \bottomrule
        \end{tabular}
      \end{table}
      
      \vspace{0.2cm}
      \begin{exampleblock}{Industry Target}
        \small MMRE $< 0.25$ and PRED(25) $> 0.75$ are considered \textit{acceptable} (Conte et al., 1986)
      \end{exampleblock}
    \end{column}
  \end{columns}
  
\end{frame}

% ==================== SLIDE 9: Overall Results ====================
\begin{frame}{Results: Model Performance Comparison}
  
  \vspace{-0.5cm}
  \begin{figure}
    \centering
    \includegraphics[width=0.72\textwidth,height=0.62\textheight,keepaspectratio]{figures/fig4_model_comparison.pdf}
  \end{figure}
  
  \vspace{-0.8cm}
  \begin{block}{Key Findings}
    \vspace{-0.1cm}
    \small
    \begin{columns}[T]
      \begin{column}{0.48\textwidth}
        \begin{itemize}
          \setlength\itemsep{-0.1em}
          \item RF reduces MMRE by \textcolor{secondarygreen}{\textbf{34\%}}
          \item PRED(25) improves to \textcolor{secondarygreen}{\textbf{58\%}}
        \end{itemize}
      \end{column}
      \begin{column}{0.48\textwidth}
        \begin{itemize}
          \setlength\itemsep{-0.1em}
          \item Ensemble methods superior to baselines
        \end{itemize}
      \end{column}
    \end{columns}
    \vspace{-0.1cm}
  \end{block}
  
\end{frame}

% ==================== SLIDE 10: Schema-Specific Performance ====================
\begin{frame}{Results by Schema: LOC vs FP vs UCP}
  
  \begin{figure}
    \centering
    \includegraphics[width=0.85\textwidth]{figures/fig5_schema_performance.pdf}
    \caption*{\textbf{Performance varies by schema due to data availability}}
  \end{figure}
  
  \vspace{-0.2cm}
  \begin{columns}[T]
    \begin{column}{0.32\textwidth}
      \colorbox{secondarygreen!20}{\parbox{0.95\textwidth}{
        \centering \textbf{LOC Schema} \\
        \small $n=180$ samples \\
        \textcolor{secondarygreen}{Stable \& reliable}
      }}
    \end{column}
    
    \begin{column}{0.32\textwidth}
      \colorbox{primaryblue!20}{\parbox{0.95\textwidth}{
        \centering \textbf{FP Schema} \\
        \small $n=95$ samples \\
        \textcolor{primaryblue}{Moderate accuracy}
      }}
    \end{column}
    
    \begin{column}{0.32\textwidth}
      \colorbox{accentorange!20}{\parbox{0.95\textwidth}{
        \centering \textbf{UCP Schema} \\
        \small $n=45$ samples \\
        \textcolor{accentorange}{Higher uncertainty}
      }}
    \end{column}
  \end{columns}
  
\end{frame}

% ==================== SLIDE 11: Error Analysis ====================
\begin{frame}{Error Analysis \& Model Interpretability}
  
  \begin{columns}[T]
    \begin{column}{0.48\textwidth}
      \begin{figure}
        \includegraphics[width=\textwidth]{figures/fig6_actual_vs_predicted.pdf}
        \caption*{\small Actual vs Predicted (RF)}
      \end{figure}
    \end{column}
    
    \begin{column}{0.48\textwidth}
      \begin{figure}
        \includegraphics[width=\textwidth]{figures/fig7_feature_importance.pdf}
        \caption*{\small Feature Importance}
      \end{figure}
    \end{column}
  \end{columns}
  
  \vspace{-0.2cm}
  \textbf{Insights:}
  \begin{itemize}
    \item Model performs well across \textit{all project sizes}
    \item \textbf{Size metric} is the dominant predictor (38\% importance)
    \item \textbf{Schema type} contributes 22\% $\rightarrow$ justifies multi-schema approach
  \end{itemize}
  
\end{frame}

% ==================== SLIDE 12: Residual Analysis ====================
\begin{frame}{Residual Analysis: Model Diagnostics}
  
  \begin{figure}
    \centering
    \includegraphics[width=0.85\textwidth]{figures/fig8_residual_analysis.pdf}
    \caption*{\textbf{Residual plots confirm model assumptions}}
  \end{figure}
  
  \vspace{-0.2cm}
  \begin{block}{Validation}
    \begin{itemize}
      \item \textbf{Random scatter} around zero $\rightarrow$ No systematic bias
      \item \textbf{Near-normal distribution} $\rightarrow$ Reliable confidence intervals
      \item Small standard deviation indicates \textcolor{secondarygreen}{consistent predictions}
    \end{itemize}
  \end{block}
  
\end{frame}

% ==================== SLIDE 13: Deployment & System Architecture ====================
\begin{frame}{Deployment: Multi-Schema Prediction API}
  
  \begin{figure}
    \centering
    \includegraphics[width=0.80\textwidth]{figures/fig9_system_architecture.pdf}
    \caption*{\textbf{REST API architecture for production deployment}}
  \end{figure}
  
  \vspace{-0.2cm}
  \textbf{Features:}
  \begin{itemize}
    \item \textbf{Schema-aware routing:} Automatic detection of LOC/FP/UCP input
    \item \textbf{Model registry:} Separate trained models per schema
    \item \textbf{Traceability:} Preserves data source \& confidence scores
    \item \textbf{Extensibility:} Ready for requirement analysis \& Jira integration
  \end{itemize}
  
\end{frame}

% ==================== SLIDE 14: Practical Applications ====================
\begin{frame}{Practical Applications \& Use Cases}
  
  \begin{columns}[T]
    \begin{column}{0.48\textwidth}
      \textbf{Current Deployment:}
      \begin{itemize}
        \item \textbf{API Endpoint:} \texttt{/api/estimate}
        \item \textbf{Input:} Project requirements or metrics
        \item \textbf{Output:} Effort, Duration, Team Size + confidence
      \end{itemize}
      
      \vspace{0.4cm}
      \textbf{Supported Modes:}
      \begin{enumerate}
        \item LOC-based estimation
        \item Function Point estimation
        \item Use Case Point estimation
        \item Mixed/automatic detection
      \end{enumerate}
    \end{column}
    
    \begin{column}{0.48\textwidth}
      \textbf{Future Extensions:}
      \begin{itemize}
        \item \textbf{NLP Integration:} Extract metrics from requirement documents
        \item \textbf{Story Point Mapping:} Agile project support
        \item \textbf{Jira Plugin:} Real-time estimation in issue tracking
        \item \textbf{Continuous Learning:} Feedback loop for model retraining
      \end{itemize}
      
      \vspace{0.3cm}
      \begin{exampleblock}{Impact}
        \small Reduces manual estimation time by \textbf{70\%} while improving accuracy by \textbf{34\%}
      \end{exampleblock}
    \end{column}
  \end{columns}
  
\end{frame}

% ==================== SLIDE 15: Limitations & Future Work ====================
\begin{frame}{Limitations \& Future Directions}
  
  \begin{columns}[T]
    \begin{column}{0.48\textwidth}
      \textbf{Current Limitations:}
      \begin{enumerate}
        \item \textcolor{alertred}{\textbf{Data scarcity}} in UCP schema \\
        \small $\rightarrow$ Higher uncertainty in predictions
        
        \item \textcolor{alertred}{\textbf{Context factors}} not fully captured \\
        \small $\rightarrow$ Domain-specific calibration needed
        
        \item \textcolor{alertred}{\textbf{Static models}} require periodic retraining \\
        \small $\rightarrow$ Technology evolution not tracked
      \end{enumerate}
      
      \vspace{0.3cm}
      \textbf{Honest Assessment:}
      \begin{itemize}
        \item Model works best for \textit{similar project types}
        \item Extreme outliers still challenging
      \end{itemize}
    \end{column}
    
    \begin{column}{0.48\textwidth}
      \textbf{Future Roadmap:}
      \begin{enumerate}
        \item \textbf{Data Augmentation:} \\
        \small Collect more UCP projects, synthetic data generation
        
        \item \textbf{Deep Learning:} \\
        \small Neural networks for complex non-linear relationships
        
        \item \textbf{Online Learning:} \\
        \small Incremental updates from project feedback
        
        \item \textbf{Multi-modal Input:} \\
        \small Combine metrics + text requirements + historical data
        
        \item \textbf{Uncertainty Quantification:} \\
        \small Probabilistic predictions with confidence intervals
      \end{enumerate}
    \end{column}
  \end{columns}
  
\end{frame}

% ==================== SLIDE 16: Conclusion ====================
\begin{frame}{Conclusion}
  
  \begin{block}{Summary of Contributions}
    \small
    \begin{enumerate}
      \item \textbf{Unified Pipeline:} Auto-normalization for LOC/FP/UCP data
      \item \textbf{Validation:} RF reduces MMRE by \textcolor{secondarygreen}{\textbf{34\%}} vs COCOMO II
      \item \textbf{Deployment:} REST API for real-world usage
    \end{enumerate}
  \end{block}
  
  \vspace{0.2cm}
  
  \begin{columns}[T]
    \begin{column}{0.48\textwidth}
      \textbf{Key Takeaways:}
      \small
      \begin{itemize}
        \item Data integration crucial
        \item Ensemble ML superior
        \item Schema-aware modeling
      \end{itemize}
    \end{column}
    
    \begin{column}{0.48\textwidth}
      \textbf{Impact:}
      \small
      \begin{itemize}
        \item Data-driven planning
        \item Reduced estimation bias
        \item Multi-schema support
      \end{itemize}
    \end{column}
  \end{columns}
  
  \vspace{0.2cm}
  
  \begin{center}
    \large \textbf{Thank you for your attention!}
    
    \vspace{0.3cm}
    \normalsize Questions \& Discussion
  \end{center}
  
\end{frame}

% ==================== BACKUP SLIDES ====================

\appendix

\begin{frame}[allowframebreaks]{References}
  \footnotesize
  \begin{thebibliography}{99}
    
    \bibitem{boehm2000} B. Boehm et al., \textit{Software Cost Estimation with COCOMO II}. Prentice Hall, 2000.
    
    \bibitem{conte1986} S. D. Conte, H. E. Dunsmore, and V. Y. Shen, \textit{Software Engineering Metrics and Models}. Benjamin-Cummings Publishing, 1986.
    
    \bibitem{wen2012} J. Wen, S. Li, Z. Lin, Y. Hu, and C. Huang, ``Systematic literature review of machine learning based software development effort estimation models,'' \textit{Information and Software Technology}, vol. 54, no. 1, pp. 41-59, 2012.
    
    \bibitem{jorgensen2007} M. Jørgensen and M. Shepperd, ``A systematic review of software development cost estimation studies,'' \textit{IEEE Transactions on Software Engineering}, vol. 33, no. 1, pp. 33-53, 2007.
    
    \bibitem{desharnais1989} J. M. Desharnais, \textit{Analyse statistique de la productivite des projets de developpement en informatique a partir de la technique des points de fonction}. Master's thesis, University of Montreal, 1989.
    
    \bibitem{albrecht1983} A. J. Albrecht and J. E. Gaffney, ``Software function, source lines of code, and development effort prediction: A software science validation,'' \textit{IEEE Transactions on Software Engineering}, vol. SE-9, no. 6, pp. 639-648, 1983.
    
    \bibitem{karner1993} G. Karner, ``Resource estimation for objectory projects,'' Objective Systems SF AB, 1993.
    
  \end{thebibliography}
\end{frame}

\begin{frame}{Backup: Detailed Metrics Table}
  
  \begin{table}[h]
    \centering
    \small
    \begin{tabular}{lcccccc}
      \toprule
      \textbf{Model} & \textbf{MAE} & \textbf{RMSE} & \textbf{MMRE} & \textbf{PRED(25)} & \textbf{R²} & \textbf{Training Time} \\
      \midrule
      COCOMO II & 28.5 & 42.7 & 0.58 & 32\% & 0.52 & N/A \\
      Linear Reg & 24.3 & 38.2 & 0.51 & 38\% & 0.61 & 0.2s \\
      Decision Tree & 21.8 & 33.5 & 0.45 & 45\% & 0.68 & 1.5s \\
      \textbf{Random Forest} & \textbf{18.4} & \textbf{27.8} & \textbf{0.38} & \textbf{58\%} & \textbf{0.78} & 8.3s \\
      Gradient Boost & 19.2 & 29.1 & 0.40 & 55\% & 0.76 & 12.1s \\
      \bottomrule
    \end{tabular}
    \caption*{Complete performance metrics on test set (n=64)}
  \end{table}
  
  \vspace{0.3cm}
  
  \textbf{Statistical Significance:}
  \begin{itemize}
    \item Paired t-test: RF vs COCOMO II, $p < 0.001$
    \item Wilcoxon signed-rank test: RF vs GB, $p = 0.042$
  \end{itemize}
  
\end{frame}

\begin{frame}{Backup: Hyperparameter Tuning}
  
  \textbf{Random Forest Optimized Parameters:}
  \begin{itemize}
    \item \texttt{n\_estimators}: 100 (tested: 50, 100, 200)
    \item \texttt{max\_depth}: 15 (tested: 10, 15, 20, None)
    \item \texttt{min\_samples\_split}: 5 (tested: 2, 5, 10)
    \item \texttt{min\_samples\_leaf}: 2 (tested: 1, 2, 4)
    \item \texttt{max\_features}: 'sqrt' (tested: 'sqrt', 'log2', None)
  \end{itemize}
  
  \vspace{0.4cm}
  
  \textbf{GridSearchCV Configuration:}
  \begin{itemize}
    \item 5-fold cross-validation
    \item Scoring metric: negative MAE
    \item Total fits: 540 (108 candidates × 5 folds)
    \item Best CV score: MAE = 19.2 PM
  \end{itemize}
  
\end{frame}

\end{document}
